\documentclass[11pt]{article}

    \usepackage[breakable]{tcolorbox}
    \usepackage{parskip} % Stop auto-indenting (to mimic markdown behaviour)
    
    \usepackage{iftex}
    \ifPDFTeX
    	\usepackage[T1]{fontenc}
    	\usepackage{mathpazo}
    \else
    	\usepackage{fontspec}
    \fi

    % Basic figure setup, for now with no caption control since it's done
    % automatically by Pandoc (which extracts ![](path) syntax from Markdown).
    \usepackage{graphicx}
    % Maintain compatibility with old templates. Remove in nbconvert 6.0
    \let\Oldincludegraphics\includegraphics
    % Ensure that by default, figures have no caption (until we provide a
    % proper Figure object with a Caption API and a way to capture that
    % in the conversion process - todo).
    \usepackage{caption}
    \DeclareCaptionFormat{nocaption}{}
    \captionsetup{format=nocaption,aboveskip=0pt,belowskip=0pt}

    \usepackage[Export]{adjustbox} % Used to constrain images to a maximum size
    \adjustboxset{max size={0.9\linewidth}{0.9\paperheight}}
    \usepackage{float}
    \floatplacement{figure}{H} % forces figures to be placed at the correct location
    \usepackage{xcolor} % Allow colors to be defined
    \usepackage{enumerate} % Needed for markdown enumerations to work
    \usepackage{geometry} % Used to adjust the document margins
    \usepackage{amsmath} % Equations
    \usepackage{amssymb} % Equations
    \usepackage{textcomp} % defines textquotesingle
    % Hack from http://tex.stackexchange.com/a/47451/13684:
    \AtBeginDocument{%
        \def\PYZsq{\textquotesingle}% Upright quotes in Pygmentized code
    }
    \usepackage{upquote} % Upright quotes for verbatim code
    \usepackage{eurosym} % defines \euro
    \usepackage[mathletters]{ucs} % Extended unicode (utf-8) support
    \usepackage{fancyvrb} % verbatim replacement that allows latex
    \usepackage{grffile} % extends the file name processing of package graphics 
                         % to support a larger range
    \makeatletter % fix for grffile with XeLaTeX
    \def\Gread@@xetex#1{%
      \IfFileExists{"\Gin@base".bb}%
      {\Gread@eps{\Gin@base.bb}}%
      {\Gread@@xetex@aux#1}%
    }
    \makeatother

    % The hyperref package gives us a pdf with properly built
    % internal navigation ('pdf bookmarks' for the table of contents,
    % internal cross-reference links, web links for URLs, etc.)
    \usepackage{hyperref}
    % The default LaTeX title has an obnoxious amount of whitespace. By default,
    % titling removes some of it. It also provides customization options.
    \usepackage{titling}
    \usepackage{longtable} % longtable support required by pandoc >1.10
    \usepackage{booktabs}  % table support for pandoc > 1.12.2
    \usepackage[inline]{enumitem} % IRkernel/repr support (it uses the enumerate* environment)
    \usepackage[normalem]{ulem} % ulem is needed to support strikethroughs (\sout)
                                % normalem makes italics be italics, not underlines
    \usepackage{mathrsfs}
    

    
    % Colors for the hyperref package
    \definecolor{urlcolor}{rgb}{0,.145,.698}
    \definecolor{linkcolor}{rgb}{.71,0.21,0.01}
    \definecolor{citecolor}{rgb}{.12,.54,.11}

    % ANSI colors
    \definecolor{ansi-black}{HTML}{3E424D}
    \definecolor{ansi-black-intense}{HTML}{282C36}
    \definecolor{ansi-red}{HTML}{E75C58}
    \definecolor{ansi-red-intense}{HTML}{B22B31}
    \definecolor{ansi-green}{HTML}{00A250}
    \definecolor{ansi-green-intense}{HTML}{007427}
    \definecolor{ansi-yellow}{HTML}{DDB62B}
    \definecolor{ansi-yellow-intense}{HTML}{B27D12}
    \definecolor{ansi-blue}{HTML}{208FFB}
    \definecolor{ansi-blue-intense}{HTML}{0065CA}
    \definecolor{ansi-magenta}{HTML}{D160C4}
    \definecolor{ansi-magenta-intense}{HTML}{A03196}
    \definecolor{ansi-cyan}{HTML}{60C6C8}
    \definecolor{ansi-cyan-intense}{HTML}{258F8F}
    \definecolor{ansi-white}{HTML}{C5C1B4}
    \definecolor{ansi-white-intense}{HTML}{A1A6B2}
    \definecolor{ansi-default-inverse-fg}{HTML}{FFFFFF}
    \definecolor{ansi-default-inverse-bg}{HTML}{000000}

    % commands and environments needed by pandoc snippets
    % extracted from the output of `pandoc -s`
    \providecommand{\tightlist}{%
      \setlength{\itemsep}{0pt}\setlength{\parskip}{0pt}}
    \DefineVerbatimEnvironment{Highlighting}{Verbatim}{commandchars=\\\{\}}
    % Add ',fontsize=\small' for more characters per line
    \newenvironment{Shaded}{}{}
    \newcommand{\KeywordTok}[1]{\textcolor[rgb]{0.00,0.44,0.13}{\textbf{{#1}}}}
    \newcommand{\DataTypeTok}[1]{\textcolor[rgb]{0.56,0.13,0.00}{{#1}}}
    \newcommand{\DecValTok}[1]{\textcolor[rgb]{0.25,0.63,0.44}{{#1}}}
    \newcommand{\BaseNTok}[1]{\textcolor[rgb]{0.25,0.63,0.44}{{#1}}}
    \newcommand{\FloatTok}[1]{\textcolor[rgb]{0.25,0.63,0.44}{{#1}}}
    \newcommand{\CharTok}[1]{\textcolor[rgb]{0.25,0.44,0.63}{{#1}}}
    \newcommand{\StringTok}[1]{\textcolor[rgb]{0.25,0.44,0.63}{{#1}}}
    \newcommand{\CommentTok}[1]{\textcolor[rgb]{0.38,0.63,0.69}{\textit{{#1}}}}
    \newcommand{\OtherTok}[1]{\textcolor[rgb]{0.00,0.44,0.13}{{#1}}}
    \newcommand{\AlertTok}[1]{\textcolor[rgb]{1.00,0.00,0.00}{\textbf{{#1}}}}
    \newcommand{\FunctionTok}[1]{\textcolor[rgb]{0.02,0.16,0.49}{{#1}}}
    \newcommand{\RegionMarkerTok}[1]{{#1}}
    \newcommand{\ErrorTok}[1]{\textcolor[rgb]{1.00,0.00,0.00}{\textbf{{#1}}}}
    \newcommand{\NormalTok}[1]{{#1}}
    
    % Additional commands for more recent versions of Pandoc
    \newcommand{\ConstantTok}[1]{\textcolor[rgb]{0.53,0.00,0.00}{{#1}}}
    \newcommand{\SpecialCharTok}[1]{\textcolor[rgb]{0.25,0.44,0.63}{{#1}}}
    \newcommand{\VerbatimStringTok}[1]{\textcolor[rgb]{0.25,0.44,0.63}{{#1}}}
    \newcommand{\SpecialStringTok}[1]{\textcolor[rgb]{0.73,0.40,0.53}{{#1}}}
    \newcommand{\ImportTok}[1]{{#1}}
    \newcommand{\DocumentationTok}[1]{\textcolor[rgb]{0.73,0.13,0.13}{\textit{{#1}}}}
    \newcommand{\AnnotationTok}[1]{\textcolor[rgb]{0.38,0.63,0.69}{\textbf{\textit{{#1}}}}}
    \newcommand{\CommentVarTok}[1]{\textcolor[rgb]{0.38,0.63,0.69}{\textbf{\textit{{#1}}}}}
    \newcommand{\VariableTok}[1]{\textcolor[rgb]{0.10,0.09,0.49}{{#1}}}
    \newcommand{\ControlFlowTok}[1]{\textcolor[rgb]{0.00,0.44,0.13}{\textbf{{#1}}}}
    \newcommand{\OperatorTok}[1]{\textcolor[rgb]{0.40,0.40,0.40}{{#1}}}
    \newcommand{\BuiltInTok}[1]{{#1}}
    \newcommand{\ExtensionTok}[1]{{#1}}
    \newcommand{\PreprocessorTok}[1]{\textcolor[rgb]{0.74,0.48,0.00}{{#1}}}
    \newcommand{\AttributeTok}[1]{\textcolor[rgb]{0.49,0.56,0.16}{{#1}}}
    \newcommand{\InformationTok}[1]{\textcolor[rgb]{0.38,0.63,0.69}{\textbf{\textit{{#1}}}}}
    \newcommand{\WarningTok}[1]{\textcolor[rgb]{0.38,0.63,0.69}{\textbf{\textit{{#1}}}}}
    
    
    % Define a nice break command that doesn't care if a line doesn't already
    % exist.
    \def\br{\hspace*{\fill} \\* }
    % Math Jax compatibility definitions
    \def\gt{>}
    \def\lt{<}
    \let\Oldtex\TeX
    \let\Oldlatex\LaTeX
    \renewcommand{\TeX}{\textrm{\Oldtex}}
    \renewcommand{\LaTeX}{\textrm{\Oldlatex}}
    % Document parameters
    % Document title
    \title{dataAnalysis}
    
    
    
    
    
% Pygments definitions
\makeatletter
\def\PY@reset{\let\PY@it=\relax \let\PY@bf=\relax%
    \let\PY@ul=\relax \let\PY@tc=\relax%
    \let\PY@bc=\relax \let\PY@ff=\relax}
\def\PY@tok#1{\csname PY@tok@#1\endcsname}
\def\PY@toks#1+{\ifx\relax#1\empty\else%
    \PY@tok{#1}\expandafter\PY@toks\fi}
\def\PY@do#1{\PY@bc{\PY@tc{\PY@ul{%
    \PY@it{\PY@bf{\PY@ff{#1}}}}}}}
\def\PY#1#2{\PY@reset\PY@toks#1+\relax+\PY@do{#2}}

\expandafter\def\csname PY@tok@w\endcsname{\def\PY@tc##1{\textcolor[rgb]{0.73,0.73,0.73}{##1}}}
\expandafter\def\csname PY@tok@c\endcsname{\let\PY@it=\textit\def\PY@tc##1{\textcolor[rgb]{0.25,0.50,0.50}{##1}}}
\expandafter\def\csname PY@tok@cp\endcsname{\def\PY@tc##1{\textcolor[rgb]{0.74,0.48,0.00}{##1}}}
\expandafter\def\csname PY@tok@k\endcsname{\let\PY@bf=\textbf\def\PY@tc##1{\textcolor[rgb]{0.00,0.50,0.00}{##1}}}
\expandafter\def\csname PY@tok@kp\endcsname{\def\PY@tc##1{\textcolor[rgb]{0.00,0.50,0.00}{##1}}}
\expandafter\def\csname PY@tok@kt\endcsname{\def\PY@tc##1{\textcolor[rgb]{0.69,0.00,0.25}{##1}}}
\expandafter\def\csname PY@tok@o\endcsname{\def\PY@tc##1{\textcolor[rgb]{0.40,0.40,0.40}{##1}}}
\expandafter\def\csname PY@tok@ow\endcsname{\let\PY@bf=\textbf\def\PY@tc##1{\textcolor[rgb]{0.67,0.13,1.00}{##1}}}
\expandafter\def\csname PY@tok@nb\endcsname{\def\PY@tc##1{\textcolor[rgb]{0.00,0.50,0.00}{##1}}}
\expandafter\def\csname PY@tok@nf\endcsname{\def\PY@tc##1{\textcolor[rgb]{0.00,0.00,1.00}{##1}}}
\expandafter\def\csname PY@tok@nc\endcsname{\let\PY@bf=\textbf\def\PY@tc##1{\textcolor[rgb]{0.00,0.00,1.00}{##1}}}
\expandafter\def\csname PY@tok@nn\endcsname{\let\PY@bf=\textbf\def\PY@tc##1{\textcolor[rgb]{0.00,0.00,1.00}{##1}}}
\expandafter\def\csname PY@tok@ne\endcsname{\let\PY@bf=\textbf\def\PY@tc##1{\textcolor[rgb]{0.82,0.25,0.23}{##1}}}
\expandafter\def\csname PY@tok@nv\endcsname{\def\PY@tc##1{\textcolor[rgb]{0.10,0.09,0.49}{##1}}}
\expandafter\def\csname PY@tok@no\endcsname{\def\PY@tc##1{\textcolor[rgb]{0.53,0.00,0.00}{##1}}}
\expandafter\def\csname PY@tok@nl\endcsname{\def\PY@tc##1{\textcolor[rgb]{0.63,0.63,0.00}{##1}}}
\expandafter\def\csname PY@tok@ni\endcsname{\let\PY@bf=\textbf\def\PY@tc##1{\textcolor[rgb]{0.60,0.60,0.60}{##1}}}
\expandafter\def\csname PY@tok@na\endcsname{\def\PY@tc##1{\textcolor[rgb]{0.49,0.56,0.16}{##1}}}
\expandafter\def\csname PY@tok@nt\endcsname{\let\PY@bf=\textbf\def\PY@tc##1{\textcolor[rgb]{0.00,0.50,0.00}{##1}}}
\expandafter\def\csname PY@tok@nd\endcsname{\def\PY@tc##1{\textcolor[rgb]{0.67,0.13,1.00}{##1}}}
\expandafter\def\csname PY@tok@s\endcsname{\def\PY@tc##1{\textcolor[rgb]{0.73,0.13,0.13}{##1}}}
\expandafter\def\csname PY@tok@sd\endcsname{\let\PY@it=\textit\def\PY@tc##1{\textcolor[rgb]{0.73,0.13,0.13}{##1}}}
\expandafter\def\csname PY@tok@si\endcsname{\let\PY@bf=\textbf\def\PY@tc##1{\textcolor[rgb]{0.73,0.40,0.53}{##1}}}
\expandafter\def\csname PY@tok@se\endcsname{\let\PY@bf=\textbf\def\PY@tc##1{\textcolor[rgb]{0.73,0.40,0.13}{##1}}}
\expandafter\def\csname PY@tok@sr\endcsname{\def\PY@tc##1{\textcolor[rgb]{0.73,0.40,0.53}{##1}}}
\expandafter\def\csname PY@tok@ss\endcsname{\def\PY@tc##1{\textcolor[rgb]{0.10,0.09,0.49}{##1}}}
\expandafter\def\csname PY@tok@sx\endcsname{\def\PY@tc##1{\textcolor[rgb]{0.00,0.50,0.00}{##1}}}
\expandafter\def\csname PY@tok@m\endcsname{\def\PY@tc##1{\textcolor[rgb]{0.40,0.40,0.40}{##1}}}
\expandafter\def\csname PY@tok@gh\endcsname{\let\PY@bf=\textbf\def\PY@tc##1{\textcolor[rgb]{0.00,0.00,0.50}{##1}}}
\expandafter\def\csname PY@tok@gu\endcsname{\let\PY@bf=\textbf\def\PY@tc##1{\textcolor[rgb]{0.50,0.00,0.50}{##1}}}
\expandafter\def\csname PY@tok@gd\endcsname{\def\PY@tc##1{\textcolor[rgb]{0.63,0.00,0.00}{##1}}}
\expandafter\def\csname PY@tok@gi\endcsname{\def\PY@tc##1{\textcolor[rgb]{0.00,0.63,0.00}{##1}}}
\expandafter\def\csname PY@tok@gr\endcsname{\def\PY@tc##1{\textcolor[rgb]{1.00,0.00,0.00}{##1}}}
\expandafter\def\csname PY@tok@ge\endcsname{\let\PY@it=\textit}
\expandafter\def\csname PY@tok@gs\endcsname{\let\PY@bf=\textbf}
\expandafter\def\csname PY@tok@gp\endcsname{\let\PY@bf=\textbf\def\PY@tc##1{\textcolor[rgb]{0.00,0.00,0.50}{##1}}}
\expandafter\def\csname PY@tok@go\endcsname{\def\PY@tc##1{\textcolor[rgb]{0.53,0.53,0.53}{##1}}}
\expandafter\def\csname PY@tok@gt\endcsname{\def\PY@tc##1{\textcolor[rgb]{0.00,0.27,0.87}{##1}}}
\expandafter\def\csname PY@tok@err\endcsname{\def\PY@bc##1{\setlength{\fboxsep}{0pt}\fcolorbox[rgb]{1.00,0.00,0.00}{1,1,1}{\strut ##1}}}
\expandafter\def\csname PY@tok@kc\endcsname{\let\PY@bf=\textbf\def\PY@tc##1{\textcolor[rgb]{0.00,0.50,0.00}{##1}}}
\expandafter\def\csname PY@tok@kd\endcsname{\let\PY@bf=\textbf\def\PY@tc##1{\textcolor[rgb]{0.00,0.50,0.00}{##1}}}
\expandafter\def\csname PY@tok@kn\endcsname{\let\PY@bf=\textbf\def\PY@tc##1{\textcolor[rgb]{0.00,0.50,0.00}{##1}}}
\expandafter\def\csname PY@tok@kr\endcsname{\let\PY@bf=\textbf\def\PY@tc##1{\textcolor[rgb]{0.00,0.50,0.00}{##1}}}
\expandafter\def\csname PY@tok@bp\endcsname{\def\PY@tc##1{\textcolor[rgb]{0.00,0.50,0.00}{##1}}}
\expandafter\def\csname PY@tok@fm\endcsname{\def\PY@tc##1{\textcolor[rgb]{0.00,0.00,1.00}{##1}}}
\expandafter\def\csname PY@tok@vc\endcsname{\def\PY@tc##1{\textcolor[rgb]{0.10,0.09,0.49}{##1}}}
\expandafter\def\csname PY@tok@vg\endcsname{\def\PY@tc##1{\textcolor[rgb]{0.10,0.09,0.49}{##1}}}
\expandafter\def\csname PY@tok@vi\endcsname{\def\PY@tc##1{\textcolor[rgb]{0.10,0.09,0.49}{##1}}}
\expandafter\def\csname PY@tok@vm\endcsname{\def\PY@tc##1{\textcolor[rgb]{0.10,0.09,0.49}{##1}}}
\expandafter\def\csname PY@tok@sa\endcsname{\def\PY@tc##1{\textcolor[rgb]{0.73,0.13,0.13}{##1}}}
\expandafter\def\csname PY@tok@sb\endcsname{\def\PY@tc##1{\textcolor[rgb]{0.73,0.13,0.13}{##1}}}
\expandafter\def\csname PY@tok@sc\endcsname{\def\PY@tc##1{\textcolor[rgb]{0.73,0.13,0.13}{##1}}}
\expandafter\def\csname PY@tok@dl\endcsname{\def\PY@tc##1{\textcolor[rgb]{0.73,0.13,0.13}{##1}}}
\expandafter\def\csname PY@tok@s2\endcsname{\def\PY@tc##1{\textcolor[rgb]{0.73,0.13,0.13}{##1}}}
\expandafter\def\csname PY@tok@sh\endcsname{\def\PY@tc##1{\textcolor[rgb]{0.73,0.13,0.13}{##1}}}
\expandafter\def\csname PY@tok@s1\endcsname{\def\PY@tc##1{\textcolor[rgb]{0.73,0.13,0.13}{##1}}}
\expandafter\def\csname PY@tok@mb\endcsname{\def\PY@tc##1{\textcolor[rgb]{0.40,0.40,0.40}{##1}}}
\expandafter\def\csname PY@tok@mf\endcsname{\def\PY@tc##1{\textcolor[rgb]{0.40,0.40,0.40}{##1}}}
\expandafter\def\csname PY@tok@mh\endcsname{\def\PY@tc##1{\textcolor[rgb]{0.40,0.40,0.40}{##1}}}
\expandafter\def\csname PY@tok@mi\endcsname{\def\PY@tc##1{\textcolor[rgb]{0.40,0.40,0.40}{##1}}}
\expandafter\def\csname PY@tok@il\endcsname{\def\PY@tc##1{\textcolor[rgb]{0.40,0.40,0.40}{##1}}}
\expandafter\def\csname PY@tok@mo\endcsname{\def\PY@tc##1{\textcolor[rgb]{0.40,0.40,0.40}{##1}}}
\expandafter\def\csname PY@tok@ch\endcsname{\let\PY@it=\textit\def\PY@tc##1{\textcolor[rgb]{0.25,0.50,0.50}{##1}}}
\expandafter\def\csname PY@tok@cm\endcsname{\let\PY@it=\textit\def\PY@tc##1{\textcolor[rgb]{0.25,0.50,0.50}{##1}}}
\expandafter\def\csname PY@tok@cpf\endcsname{\let\PY@it=\textit\def\PY@tc##1{\textcolor[rgb]{0.25,0.50,0.50}{##1}}}
\expandafter\def\csname PY@tok@c1\endcsname{\let\PY@it=\textit\def\PY@tc##1{\textcolor[rgb]{0.25,0.50,0.50}{##1}}}
\expandafter\def\csname PY@tok@cs\endcsname{\let\PY@it=\textit\def\PY@tc##1{\textcolor[rgb]{0.25,0.50,0.50}{##1}}}

\def\PYZbs{\char`\\}
\def\PYZus{\char`\_}
\def\PYZob{\char`\{}
\def\PYZcb{\char`\}}
\def\PYZca{\char`\^}
\def\PYZam{\char`\&}
\def\PYZlt{\char`\<}
\def\PYZgt{\char`\>}
\def\PYZsh{\char`\#}
\def\PYZpc{\char`\%}
\def\PYZdl{\char`\$}
\def\PYZhy{\char`\-}
\def\PYZsq{\char`\'}
\def\PYZdq{\char`\"}
\def\PYZti{\char`\~}
% for compatibility with earlier versions
\def\PYZat{@}
\def\PYZlb{[}
\def\PYZrb{]}
\makeatother


    % For linebreaks inside Verbatim environment from package fancyvrb. 
    \makeatletter
        \newbox\Wrappedcontinuationbox 
        \newbox\Wrappedvisiblespacebox 
        \newcommand*\Wrappedvisiblespace {\textcolor{red}{\textvisiblespace}} 
        \newcommand*\Wrappedcontinuationsymbol {\textcolor{red}{\llap{\tiny$\m@th\hookrightarrow$}}} 
        \newcommand*\Wrappedcontinuationindent {3ex } 
        \newcommand*\Wrappedafterbreak {\kern\Wrappedcontinuationindent\copy\Wrappedcontinuationbox} 
        % Take advantage of the already applied Pygments mark-up to insert 
        % potential linebreaks for TeX processing. 
        %        {, <, #, %, $, ' and ": go to next line. 
        %        _, }, ^, &, >, - and ~: stay at end of broken line. 
        % Use of \textquotesingle for straight quote. 
        \newcommand*\Wrappedbreaksatspecials {% 
            \def\PYGZus{\discretionary{\char`\_}{\Wrappedafterbreak}{\char`\_}}% 
            \def\PYGZob{\discretionary{}{\Wrappedafterbreak\char`\{}{\char`\{}}% 
            \def\PYGZcb{\discretionary{\char`\}}{\Wrappedafterbreak}{\char`\}}}% 
            \def\PYGZca{\discretionary{\char`\^}{\Wrappedafterbreak}{\char`\^}}% 
            \def\PYGZam{\discretionary{\char`\&}{\Wrappedafterbreak}{\char`\&}}% 
            \def\PYGZlt{\discretionary{}{\Wrappedafterbreak\char`\<}{\char`\<}}% 
            \def\PYGZgt{\discretionary{\char`\>}{\Wrappedafterbreak}{\char`\>}}% 
            \def\PYGZsh{\discretionary{}{\Wrappedafterbreak\char`\#}{\char`\#}}% 
            \def\PYGZpc{\discretionary{}{\Wrappedafterbreak\char`\%}{\char`\%}}% 
            \def\PYGZdl{\discretionary{}{\Wrappedafterbreak\char`\$}{\char`\$}}% 
            \def\PYGZhy{\discretionary{\char`\-}{\Wrappedafterbreak}{\char`\-}}% 
            \def\PYGZsq{\discretionary{}{\Wrappedafterbreak\textquotesingle}{\textquotesingle}}% 
            \def\PYGZdq{\discretionary{}{\Wrappedafterbreak\char`\"}{\char`\"}}% 
            \def\PYGZti{\discretionary{\char`\~}{\Wrappedafterbreak}{\char`\~}}% 
        } 
        % Some characters . , ; ? ! / are not pygmentized. 
        % This macro makes them "active" and they will insert potential linebreaks 
        \newcommand*\Wrappedbreaksatpunct {% 
            \lccode`\~`\.\lowercase{\def~}{\discretionary{\hbox{\char`\.}}{\Wrappedafterbreak}{\hbox{\char`\.}}}% 
            \lccode`\~`\,\lowercase{\def~}{\discretionary{\hbox{\char`\,}}{\Wrappedafterbreak}{\hbox{\char`\,}}}% 
            \lccode`\~`\;\lowercase{\def~}{\discretionary{\hbox{\char`\;}}{\Wrappedafterbreak}{\hbox{\char`\;}}}% 
            \lccode`\~`\:\lowercase{\def~}{\discretionary{\hbox{\char`\:}}{\Wrappedafterbreak}{\hbox{\char`\:}}}% 
            \lccode`\~`\?\lowercase{\def~}{\discretionary{\hbox{\char`\?}}{\Wrappedafterbreak}{\hbox{\char`\?}}}% 
            \lccode`\~`\!\lowercase{\def~}{\discretionary{\hbox{\char`\!}}{\Wrappedafterbreak}{\hbox{\char`\!}}}% 
            \lccode`\~`\/\lowercase{\def~}{\discretionary{\hbox{\char`\/}}{\Wrappedafterbreak}{\hbox{\char`\/}}}% 
            \catcode`\.\active
            \catcode`\,\active 
            \catcode`\;\active
            \catcode`\:\active
            \catcode`\?\active
            \catcode`\!\active
            \catcode`\/\active 
            \lccode`\~`\~ 	
        }
    \makeatother

    \let\OriginalVerbatim=\Verbatim
    \makeatletter
    \renewcommand{\Verbatim}[1][1]{%
        %\parskip\z@skip
        \sbox\Wrappedcontinuationbox {\Wrappedcontinuationsymbol}%
        \sbox\Wrappedvisiblespacebox {\FV@SetupFont\Wrappedvisiblespace}%
        \def\FancyVerbFormatLine ##1{\hsize\linewidth
            \vtop{\raggedright\hyphenpenalty\z@\exhyphenpenalty\z@
                \doublehyphendemerits\z@\finalhyphendemerits\z@
                \strut ##1\strut}%
        }%
        % If the linebreak is at a space, the latter will be displayed as visible
        % space at end of first line, and a continuation symbol starts next line.
        % Stretch/shrink are however usually zero for typewriter font.
        \def\FV@Space {%
            \nobreak\hskip\z@ plus\fontdimen3\font minus\fontdimen4\font
            \discretionary{\copy\Wrappedvisiblespacebox}{\Wrappedafterbreak}
            {\kern\fontdimen2\font}%
        }%
        
        % Allow breaks at special characters using \PYG... macros.
        \Wrappedbreaksatspecials
        % Breaks at punctuation characters . , ; ? ! and / need catcode=\active 	
        \OriginalVerbatim[#1,codes*=\Wrappedbreaksatpunct]%
    }
    \makeatother

    % Exact colors from NB
    \definecolor{incolor}{HTML}{303F9F}
    \definecolor{outcolor}{HTML}{D84315}
    \definecolor{cellborder}{HTML}{CFCFCF}
    \definecolor{cellbackground}{HTML}{F7F7F7}
    
    % prompt
    \makeatletter
    \newcommand{\boxspacing}{\kern\kvtcb@left@rule\kern\kvtcb@boxsep}
    \makeatother
    \newcommand{\prompt}[4]{
        \ttfamily\llap{{\color{#2}[#3]:\hspace{3pt}#4}}\vspace{-\baselineskip}
    }
    

    
    % Prevent overflowing lines due to hard-to-break entities
    \sloppy 
    % Setup hyperref package
    \hypersetup{
      breaklinks=true,  % so long urls are correctly broken across lines
      colorlinks=true,
      urlcolor=urlcolor,
      linkcolor=linkcolor,
      citecolor=citecolor,
      }
    % Slightly bigger margins than the latex defaults
    
    \geometry{verbose,tmargin=1in,bmargin=1in,lmargin=1in,rmargin=1in}
    
    

\begin{document}
    
    \maketitle
    
    

    
    Problem 2: Data Analysis

     Throughout this part I will use the library Natural Language Toolkit
(nltk), which is a powerful library used in NLP. To start I need to
download the nltk's modules that I need. 

    \begin{tcolorbox}[breakable, size=fbox, boxrule=1pt, pad at break*=1mm,colback=cellbackground, colframe=cellborder]
\prompt{In}{incolor}{ }{\boxspacing}
\begin{Verbatim}[commandchars=\\\{\}]
\PY{k+kn}{import} \PY{n+nn}{nltk}
\PY{n}{nltk}\PY{o}{.}\PY{n}{download}\PY{p}{(}\PY{l+s+s1}{\PYZsq{}}\PY{l+s+s1}{punkt}\PY{l+s+s1}{\PYZsq{}}\PY{p}{)}
\PY{n}{nltk}\PY{o}{.}\PY{n}{download}\PY{p}{(}\PY{l+s+s1}{\PYZsq{}}\PY{l+s+s1}{averaged\PYZus{}perceptron\PYZus{}tagger}\PY{l+s+s1}{\PYZsq{}}\PY{p}{)} \PY{c+c1}{\PYZsh{}used in excluding the proper nouns }
\end{Verbatim}
\end{tcolorbox}

    \begin{tcolorbox}[breakable, size=fbox, boxrule=1pt, pad at break*=1mm,colback=cellbackground, colframe=cellborder]
\prompt{In}{incolor}{177}{\boxspacing}
\begin{Verbatim}[commandchars=\\\{\}]
\PY{k+kn}{from} \PY{n+nn}{nltk}\PY{n+nn}{.}\PY{n+nn}{tokenize} \PY{k+kn}{import} \PY{n}{sent\PYZus{}tokenize}\PY{p}{,} \PY{n}{word\PYZus{}tokenize} 
\PY{k+kn}{from} \PY{n+nn}{nltk}\PY{n+nn}{.}\PY{n+nn}{tag} \PY{k+kn}{import} \PY{n}{pos\PYZus{}tag}
\PY{k+kn}{from} \PY{n+nn}{matplotlib} \PY{k+kn}{import} \PY{n}{pyplot} \PY{k}{as} \PY{n}{plt}
\PY{k+kn}{import} \PY{n+nn}{numpy} \PY{k}{as} \PY{n+nn}{np}
\PY{k+kn}{import} \PY{n+nn}{string}
\end{Verbatim}
\end{tcolorbox}

     The class Text\_Parser will be used as helper to perform the main
processing tasks on each book. For example, at initializaiton the text
in the book is parsed into words and sentences.

We define a sentence as the sequence of words that are separated by
point ``.''\\
In similar manner, a word in a text is defined as a connected sequence
of character. By connected, we mean that the characters are not
separated by a space. 

    \begin{tcolorbox}[breakable, size=fbox, boxrule=1pt, pad at break*=1mm,colback=cellbackground, colframe=cellborder]
\prompt{In}{incolor}{204}{\boxspacing}
\begin{Verbatim}[commandchars=\\\{\}]
\PY{k}{class} \PY{n+nc}{Text\PYZus{}Parser}\PY{p}{:}
    \PY{c+c1}{\PYZsh{}This class is a helper to parse each book}
    \PY{k}{def} \PY{n+nf+fm}{\PYZus{}\PYZus{}init\PYZus{}\PYZus{}}\PY{p}{(}\PY{n+nb+bp}{self}\PY{p}{,}\PY{n}{title}\PY{p}{)}\PY{p}{:}
        \PY{n}{file\PYZus{}path} \PY{o}{=} \PY{l+s+s2}{\PYZdq{}}\PY{l+s+s2}{data/}\PY{l+s+si}{\PYZob{}\PYZcb{}}\PY{l+s+s2}{.txt}\PY{l+s+s2}{\PYZdq{}}\PY{o}{.}\PY{n}{format}\PY{p}{(}\PY{n}{title}\PY{p}{)}
        \PY{n}{file\PYZus{}content} \PY{o}{=} \PY{n+nb}{open}\PY{p}{(}\PY{n}{file\PYZus{}path}\PY{p}{)}\PY{o}{.}\PY{n}{read}\PY{p}{(}\PY{p}{)}
        \PY{n+nb+bp}{self}\PY{o}{.}\PY{n}{words} \PY{o}{=} \PY{n}{nltk}\PY{o}{.}\PY{n}{word\PYZus{}tokenize}\PY{p}{(}\PY{n}{file\PYZus{}content}\PY{p}{)}
        \PY{n+nb+bp}{self}\PY{o}{.}\PY{n}{sentences} \PY{o}{=} \PY{n}{nltk}\PY{o}{.}\PY{n}{sent\PYZus{}tokenize}\PY{p}{(}\PY{n}{file\PYZus{}content}\PY{p}{)}
        
        \PY{n+nb+bp}{self}\PY{o}{.}\PY{n}{title} \PY{o}{=} \PY{n}{title}
        
        \PY{n+nb+bp}{self}\PY{o}{.}\PY{n}{sentence\PYZus{}len} \PY{o}{=} \PY{p}{[}\PY{n+nb}{len}\PY{p}{(}\PY{n}{nltk}\PY{o}{.}\PY{n}{word\PYZus{}tokenize}\PY{p}{(}\PY{n}{sentence}\PY{p}{)}\PY{p}{)} \PY{k}{for} \PY{n}{sentence} \PY{o+ow}{in} \PY{n+nb+bp}{self}\PY{o}{.}\PY{n}{sentences} \PY{p}{]}
        \PY{n+nb+bp}{self}\PY{o}{.}\PY{n}{words\PYZus{}len} \PY{o}{=} \PY{p}{[}\PY{n+nb}{len}\PY{p}{(}\PY{n}{word}\PY{p}{)} \PY{k}{for} \PY{n}{word} \PY{o+ow}{in} \PY{n+nb+bp}{self}\PY{o}{.}\PY{n}{words}\PY{p}{]}
        
        \PY{n+nb}{print}\PY{p}{(}\PY{l+s+s2}{\PYZdq{}}\PY{l+s+si}{\PYZob{}\PYZcb{}}\PY{l+s+s2}{: Num of words }\PY{l+s+si}{\PYZob{}\PYZcb{}}\PY{l+s+s2}{\PYZdq{}}\PY{o}{.}\PY{n}{format}\PY{p}{(}\PY{n}{title}\PY{p}{,}\PY{n+nb}{len}\PY{p}{(}\PY{n+nb+bp}{self}\PY{o}{.}\PY{n}{words}\PY{p}{)}\PY{p}{)}\PY{p}{)}
        \PY{n+nb}{print}\PY{p}{(}\PY{l+s+s2}{\PYZdq{}}\PY{l+s+si}{\PYZob{}\PYZcb{}}\PY{l+s+s2}{: Num of sentences }\PY{l+s+si}{\PYZob{}\PYZcb{}}\PY{l+s+s2}{\PYZdq{}}\PY{o}{.}\PY{n}{format}\PY{p}{(}\PY{n}{title}\PY{p}{,}\PY{n+nb}{len}\PY{p}{(}\PY{n+nb+bp}{self}\PY{o}{.}\PY{n}{sentences}\PY{p}{)}\PY{p}{)}\PY{p}{)}
    
    \PY{k}{def} \PY{n+nf}{plot\PYZus{}hists}\PY{p}{(}\PY{n+nb+bp}{self}\PY{p}{)}\PY{p}{:}
        \PY{n}{fig}\PY{p}{,}\PY{p}{(}\PY{n}{ax1}\PY{p}{,}\PY{n}{ax2}\PY{p}{)} \PY{o}{=} \PY{n}{plt}\PY{o}{.}\PY{n}{subplots}\PY{p}{(}\PY{l+m+mi}{1}\PY{p}{,}\PY{l+m+mi}{2}\PY{p}{)}
        \PY{n}{ax1}\PY{o}{.}\PY{n}{hist}\PY{p}{(}\PY{n+nb+bp}{self}\PY{o}{.}\PY{n}{sentence\PYZus{}len}\PY{p}{)}
        \PY{n}{ax1}\PY{o}{.}\PY{n}{set\PYZus{}title}\PY{p}{(}\PY{l+s+s2}{\PYZdq{}}\PY{l+s+s2}{Histogram of the length of sentences in }\PY{l+s+si}{\PYZob{}\PYZcb{}}\PY{l+s+s2}{\PYZdq{}}\PY{o}{.}\PY{n}{format}\PY{p}{(}\PY{n+nb+bp}{self}\PY{o}{.}\PY{n}{title}\PY{p}{)}\PY{p}{)}
        \PY{n}{fig}\PY{o}{.}\PY{n}{subplots\PYZus{}adjust}\PY{p}{(}\PY{n}{left}\PY{o}{=}\PY{l+m+mi}{0}\PY{p}{,}\PY{n}{right}\PY{o}{=}\PY{l+m+mi}{2}\PY{p}{)}
        \PY{n}{ax2}\PY{o}{.}\PY{n}{hist}\PY{p}{(}\PY{n+nb+bp}{self}\PY{o}{.}\PY{n}{words\PYZus{}len}\PY{p}{)}
        \PY{n}{ax2}\PY{o}{.}\PY{n}{set\PYZus{}title}\PY{p}{(}\PY{l+s+s2}{\PYZdq{}}\PY{l+s+s2}{Histogram of the length of words in }\PY{l+s+si}{\PYZob{}\PYZcb{}}\PY{l+s+s2}{ /n}\PY{l+s+s2}{\PYZdq{}}\PY{o}{.}\PY{n}{format}\PY{p}{(}\PY{n+nb+bp}{self}\PY{o}{.}\PY{n}{title}\PY{p}{)}\PY{p}{)}
        \PY{n}{fig}\PY{o}{.}\PY{n}{show}\PY{p}{(}\PY{p}{)}
        
    \PY{k}{def} \PY{n+nf}{exclude\PYZus{}proper\PYZus{}nouns}\PY{p}{(}\PY{n+nb+bp}{self}\PY{p}{,}\PY{n}{words}\PY{p}{)}\PY{p}{:}
        \PY{c+c1}{\PYZsh{}This method allows to remove the proper nouns in any list of words}
        \PY{n}{tagged\PYZus{}sent} \PY{o}{=} \PY{n}{pos\PYZus{}tag}\PY{p}{(}\PY{n}{words}\PY{p}{)}
        \PY{k}{return} \PY{p}{[}\PY{n}{word} \PY{k}{for} \PY{n}{word}\PY{p}{,}\PY{n}{pos} \PY{o+ow}{in} \PY{n}{tagged\PYZus{}sent} \PY{k}{if} \PY{n}{pos} \PY{o}{==} \PY{l+s+s1}{\PYZsq{}}\PY{l+s+s1}{NNP}\PY{l+s+s1}{\PYZsq{}}\PY{p}{]}
    
    \PY{k}{def} \PY{n+nf}{get\PYZus{}unique\PYZus{}words}\PY{p}{(}\PY{n+nb+bp}{self}\PY{p}{,} \PY{n}{exlude\PYZus{}proper\PYZus{}nouns} \PY{o}{=} \PY{k+kc}{True}\PY{p}{)}\PY{p}{:}
        \PY{c+c1}{\PYZsh{}REturn a list of unique words in each book (vocabulary)}
        \PY{n}{words} \PY{o}{=} \PY{n+nb}{list}\PY{p}{(}\PY{n+nb}{set}\PY{p}{(}\PY{n+nb+bp}{self}\PY{o}{.}\PY{n}{words}\PY{p}{)}\PY{p}{)}
        \PY{k}{if} \PY{n}{exlude\PYZus{}proper\PYZus{}nouns}\PY{p}{:}
            \PY{k}{return} \PY{n+nb+bp}{self}\PY{o}{.}\PY{n}{exclude\PYZus{}proper\PYZus{}nouns}\PY{p}{(}\PY{n}{words}\PY{p}{)}
        \PY{k}{return} \PY{n}{words}
        
\end{Verbatim}
\end{tcolorbox}

     Let instanciate our parser for each book and see how many words and
sentences are there in each of them. 

    \begin{tcolorbox}[breakable, size=fbox, boxrule=1pt, pad at break*=1mm,colback=cellbackground, colframe=cellborder]
\prompt{In}{incolor}{63}{\boxspacing}
\begin{Verbatim}[commandchars=\\\{\}]
\PY{n}{books} \PY{o}{=} \PY{p}{[}\PY{l+s+s2}{\PYZdq{}}\PY{l+s+s2}{Great Expectations}\PY{l+s+s2}{\PYZdq{}}\PY{p}{,}
         \PY{l+s+s2}{\PYZdq{}}\PY{l+s+s2}{Pride\PYZus{}and\PYZus{}Prejudice}\PY{l+s+s2}{\PYZdq{}}\PY{p}{,}
         \PY{l+s+s2}{\PYZdq{}}\PY{l+s+s2}{Pygmalion by Bernard Shaw}\PY{l+s+s2}{\PYZdq{}}\PY{p}{,}
         \PY{l+s+s2}{\PYZdq{}}\PY{l+s+s2}{The Brothers Karamazov by Fyodor Dostoyevsky}\PY{l+s+s2}{\PYZdq{}}\PY{p}{,}
         \PY{l+s+s2}{\PYZdq{}}\PY{l+s+s2}{The\PYZus{}Adventures\PYZus{}of\PYZus{}Tom\PYZus{}Sawyer}\PY{l+s+s2}{\PYZdq{}}\PY{p}{,}
         \PY{l+s+s2}{\PYZdq{}}\PY{l+s+s2}{Treasure Island by Robert Louis Stevenson}\PY{l+s+s2}{\PYZdq{}}\PY{p}{]}
         

\PY{n}{parsed\PYZus{}books} \PY{o}{=} \PY{p}{[}\PY{n}{Text\PYZus{}Parser}\PY{p}{(}\PY{n}{title}\PY{p}{)} \PY{k}{for} \PY{n}{title} \PY{o+ow}{in} \PY{n}{books}\PY{p}{]}
\end{Verbatim}
\end{tcolorbox}

    \begin{Verbatim}[commandchars=\\\{\}]
Great Expectations: Num of words 228938
Great Expectations: Num of sentences 7245
Pride\_and\_Prejudice: Num of words 147835
Pride\_and\_Prejudice: Num of sentences 5973
Pygmalion by Bernard Shaw: Num of words 45287
Pygmalion by Bernard Shaw: Num of sentences 3691
The Brothers Karamazov by Fyodor Dostoyevsky: Num of words 439140
The Brothers Karamazov by Fyodor Dostoyevsky: Num of sentences 19734
The\_Adventures\_of\_Tom\_Sawyer: Num of words 75361
The\_Adventures\_of\_Tom\_Sawyer: Num of sentences 2616
Treasure Island by Robert Louis Stevenson: Num of words 87655
Treasure Island by Robert Louis Stevenson: Num of sentences 3848
    \end{Verbatim}

    \hypertarget{part-a-and-b-the-histograms-of-the-length-of-words-and-sentences-used-in-each-text}{%
\subsection{Part A and B: The histograms of the length of words and
sentences used in each
text}\label{part-a-and-b-the-histograms-of-the-length-of-words-and-sentences-used-in-each-text}}

    \begin{tcolorbox}[breakable, size=fbox, boxrule=1pt, pad at break*=1mm,colback=cellbackground, colframe=cellborder]
\prompt{In}{incolor}{64}{\boxspacing}
\begin{Verbatim}[commandchars=\\\{\}]
\PY{k}{for} \PY{n}{book} \PY{o+ow}{in} \PY{n}{parsed\PYZus{}books}\PY{p}{:}
    \PY{n}{book}\PY{o}{.}\PY{n}{plot\PYZus{}hists}\PY{p}{(}\PY{p}{)}
\end{Verbatim}
\end{tcolorbox}

    \begin{Verbatim}[commandchars=\\\{\}]
/Users/massimacbookpro/opt/anaconda3/lib/python3.7/site-
packages/ipykernel\_launcher.py:23: UserWarning: Matplotlib is currently using
module://ipykernel.pylab.backend\_inline, which is a non-GUI backend, so cannot
show the figure.
    \end{Verbatim}

    \begin{center}
    \adjustimage{max size={0.9\linewidth}{0.9\paperheight}}{output_9_1.png}
    \end{center}
    { \hspace*{\fill} \\}
    
    \begin{center}
    \adjustimage{max size={0.9\linewidth}{0.9\paperheight}}{output_9_2.png}
    \end{center}
    { \hspace*{\fill} \\}
    
    \begin{center}
    \adjustimage{max size={0.9\linewidth}{0.9\paperheight}}{output_9_3.png}
    \end{center}
    { \hspace*{\fill} \\}
    
    \begin{center}
    \adjustimage{max size={0.9\linewidth}{0.9\paperheight}}{output_9_4.png}
    \end{center}
    { \hspace*{\fill} \\}
    
    \begin{center}
    \adjustimage{max size={0.9\linewidth}{0.9\paperheight}}{output_9_5.png}
    \end{center}
    { \hspace*{\fill} \\}
    
    \begin{center}
    \adjustimage{max size={0.9\linewidth}{0.9\paperheight}}{output_9_6.png}
    \end{center}
    { \hspace*{\fill} \\}
    
    \hypertarget{part-c-the-list-of-unique-words-used-in-each-book-that-is-not-used-in-any-of-the-other-books-with-the-constraint-of-excluding-proper-nouns.}{%
\subsection{Part C: The list of unique words used in each book that is
not used in any of the other books, with the constraint of excluding
proper
nouns.}\label{part-c-the-list-of-unique-words-used-in-each-book-that-is-not-used-in-any-of-the-other-books-with-the-constraint-of-excluding-proper-nouns.}}

    To exclude the proper nouns in the list of words in each book, we use
\texttt{nltk.tag.pos\_tag}, in the method
\texttt{exclude\_proper\_nouns(self,words):} in the class
\texttt{class\ Text\_Parser}, defined above.

\texttt{nltk.tag} is an interface for tagging each token in a sentence
with supplementary information, such as its part of speech. It uses
pre-trained models to classify each token. 

    \begin{tcolorbox}[breakable, size=fbox, boxrule=1pt, pad at break*=1mm,colback=cellbackground, colframe=cellborder]
\prompt{In}{incolor}{203}{\boxspacing}
\begin{Verbatim}[commandchars=\\\{\}]
\PY{n}{unique\PYZus{}words\PYZus{}bucket} \PY{o}{=} \PY{p}{[}\PY{n}{book}\PY{o}{.}\PY{n}{get\PYZus{}unique\PYZus{}words}\PY{p}{(}\PY{p}{)} \PY{k}{for} \PY{n}{book} \PY{o+ow}{in} \PY{n}{parsed\PYZus{}books}\PY{p}{]} \PY{c+c1}{\PYZsh{}get the unique words of each book}

\PY{c+c1}{\PYZsh{}Put everything to lower case}

\PY{n}{unique\PYZus{}words\PYZus{}bucket} \PY{o}{=} \PY{p}{[}\PY{p}{[}\PY{n}{word}\PY{o}{.}\PY{n}{lower}\PY{p}{(}\PY{p}{)} \PY{k}{for} \PY{n}{word} \PY{o+ow}{in} \PY{n}{u\PYZus{}words}\PY{p}{]} \PY{k}{for} \PY{n}{u\PYZus{}words} \PY{o+ow}{in} \PY{n}{unique\PYZus{}words\PYZus{}bucket}\PY{p}{]}
\PY{n}{unique\PYZus{}words\PYZus{}compared\PYZus{}to\PYZus{}other\PYZus{}books} \PY{o}{=} \PY{p}{[}\PY{p}{]}

\PY{k}{for} \PY{n}{idx}\PY{p}{,} \PY{n}{unique\PYZus{}words} \PY{o+ow}{in} \PY{n+nb}{enumerate}\PY{p}{(}\PY{n}{unique\PYZus{}words\PYZus{}bucket}\PY{p}{)}\PY{p}{:}
    
    \PY{c+c1}{\PYZsh{}Construct list that contains the unique words of the other books}
    \PY{n}{u\PYZus{}words\PYZus{}in\PYZus{}otherbooks} \PY{o}{=} \PY{n+nb}{sum}\PY{p}{(}\PY{p}{(}\PY{n}{u\PYZus{}words} \PY{k}{for} \PY{n}{u\PYZus{}words} \PY{o+ow}{in} \PY{n}{unique\PYZus{}words\PYZus{}bucket} \PY{k}{if} \PY{n}{u\PYZus{}words} \PY{o}{!=} \PY{n}{unique\PYZus{}words}\PY{p}{)}\PY{p}{,}\PY{p}{[}\PY{p}{]}\PY{p}{)}
    \PY{c+c1}{\PYZsh{}Now construct exclusive words for each book}
    \PY{n}{unique\PYZus{}words\PYZus{}compared\PYZus{}to\PYZus{}other\PYZus{}books}\PY{o}{.}\PY{n}{append}\PY{p}{(}\PY{p}{[}\PY{n}{word} \PY{k}{for} \PY{n}{word} \PY{o+ow}{in} \PY{n}{unique\PYZus{}words} \PY{k}{if} \PY{n}{word} \PY{o+ow}{not} \PY{o+ow}{in} \PY{n}{u\PYZus{}words\PYZus{}in\PYZus{}otherbooks} \PY{p}{]}\PY{p}{)}
    
    
\PY{k}{for} \PY{n}{idx}\PY{p}{,} \PY{n}{unique\PYZus{}words} \PY{o+ow}{in} \PY{n+nb}{enumerate}\PY{p}{(}\PY{n}{unique\PYZus{}words\PYZus{}bucket}\PY{p}{)}\PY{p}{:}
    \PY{n+nb}{print}\PY{p}{(}\PY{l+s+s2}{\PYZdq{}}\PY{l+s+s2}{Book: }\PY{l+s+si}{\PYZob{}\PYZcb{}}\PY{l+s+s2}{: Num of unique words }\PY{l+s+si}{\PYZob{}\PYZcb{}}\PY{l+s+s2}{. Unique words compared to other books \PYZhy{}\PYZhy{}\PYZgt{}}\PY{l+s+si}{\PYZob{}\PYZcb{}}\PY{l+s+s2}{\PYZlt{}\PYZhy{}\PYZhy{}}\PY{l+s+s2}{\PYZdq{}}\PY{o}{.}\PY{n}{format}\PY{p}{(}\PY{n}{books}\PY{p}{[}\PY{n}{idx}\PY{p}{]}\PY{p}{,}\PY{n+nb}{len}\PY{p}{(}\PY{n}{unique\PYZus{}words}\PY{p}{)}\PY{p}{,}\PY{n+nb}{len}\PY{p}{(}\PY{n}{unique\PYZus{}words\PYZus{}compared\PYZus{}to\PYZus{}other\PYZus{}books}\PY{p}{[}\PY{n}{idx}\PY{p}{]}\PY{p}{)}\PY{p}{)}\PY{p}{)}
    
\end{Verbatim}
\end{tcolorbox}

    \begin{Verbatim}[commandchars=\\\{\}]
Book: Great Expectations: Num of unique words 1542. Unique words compared to
other books -->829<--
Book: Pride\_and\_Prejudice: Num of unique words 713. Unique words compared to
other books -->273<--
Book: Pygmalion by Bernard Shaw: Num of unique words 747. Unique words compared
to other books -->311<--
Book: The Brothers Karamazov by Fyodor Dostoyevsky: Num of unique words 1920.
Unique words compared to other books -->1235<--
Book: The\_Adventures\_of\_Tom\_Sawyer: Num of unique words 1525. Unique words
compared to other books -->853<--
Book: Treasure Island by Robert Louis Stevenson: Num of unique words 772. Unique
words compared to other books -->279<--
    \end{Verbatim}

    \hypertarget{comment}{%
\subparagraph{Comment:}\label{comment}}

The algorithm above, construc first, a list of unique words of each
book. Then, each unique word is compared to the other unique words of
the other books, and we keep only the exclusive ones, that is the words
that are not in the other books.

The complexity of this algorithm is
\(O(\sum_{i,j\neq j}^k n_in_j) \sim O(kn^2)\), where \(k\) is the number
of books. The algorithm can be improved by using hash-tables and we can
achieve a complexity of \(O(kn\log n)\). 

    \hypertarget{part-d-the-longest-palindromic-sequence-in-each-text}{%
\subsection{Part D: The longest palindromic sequence in each
text}\label{part-d-the-longest-palindromic-sequence-in-each-text}}

    \begin{tcolorbox}[breakable, size=fbox, boxrule=1pt, pad at break*=1mm,colback=cellbackground, colframe=cellborder]
\prompt{In}{incolor}{158}{\boxspacing}
\begin{Verbatim}[commandchars=\\\{\}]
\PY{c+c1}{\PYZsh{}This function return the longest palindrom in a string }

\PY{k}{def} \PY{n+nf}{get\PYZus{}longest\PYZus{}palindromes}\PY{p}{(}\PY{n}{strng}\PY{p}{)}\PY{p}{:}
    \PY{n}{N} \PY{o}{=} \PY{n+nb}{len}\PY{p}{(}\PY{n}{strng}\PY{p}{)}
    \PY{n}{cache} \PY{o}{=} \PY{p}{[}\PY{p}{[}\PY{k+kc}{None}\PY{p}{]} \PY{o}{*} \PY{n}{N} \PY{k}{for} \PY{n}{\PYZus{}} \PY{o+ow}{in} \PY{n+nb}{range}\PY{p}{(}\PY{n}{N}\PY{p}{)}\PY{p}{]}

    \PY{k}{def} \PY{n+nf}{is\PYZus{}palindrome}\PY{p}{(}\PY{n}{lo}\PY{p}{,} \PY{n}{hi}\PY{p}{)}\PY{p}{:}
        \PY{k}{if} \PY{n}{cache}\PY{p}{[}\PY{n}{lo}\PY{p}{]}\PY{p}{[}\PY{n}{hi}\PY{p}{]} \PY{o+ow}{is} \PY{o+ow}{not} \PY{k+kc}{None}\PY{p}{:}
            \PY{k}{return} \PY{n}{cache}\PY{p}{[}\PY{n}{lo}\PY{p}{]}\PY{p}{[}\PY{n}{hi}\PY{p}{]}

        \PY{k}{if} \PY{n}{lo} \PY{o}{==} \PY{n}{hi}\PY{p}{:}
            \PY{k}{return} \PY{k+kc}{True}
        \PY{k}{elif} \PY{n}{lo} \PY{o}{+} \PY{l+m+mi}{1} \PY{o}{==} \PY{n}{hi}\PY{p}{:}
            \PY{k}{return} \PY{n}{strng}\PY{p}{[}\PY{n}{lo}\PY{p}{]} \PY{o}{==} \PY{n}{strng}\PY{p}{[}\PY{n}{hi}\PY{p}{]}

        \PY{n}{ans} \PY{o}{=} \PY{k+kc}{False} \PY{k}{if} \PY{n}{strng}\PY{p}{[}\PY{n}{lo}\PY{p}{]} \PY{o}{!=} \PY{n}{strng}\PY{p}{[}\PY{n}{hi}\PY{p}{]} \PY{k}{else} \PY{n}{is\PYZus{}palindrome}\PY{p}{(}\PY{n}{lo}\PY{o}{+}\PY{l+m+mi}{1}\PY{p}{,} \PY{n}{hi}\PY{o}{\PYZhy{}}\PY{l+m+mi}{1}\PY{p}{)}
        \PY{n}{cache}\PY{p}{[}\PY{n}{lo}\PY{p}{]}\PY{p}{[}\PY{n}{hi}\PY{p}{]} \PY{o}{=} \PY{n}{ans}
        \PY{k}{return} \PY{n}{ans}

    \PY{k}{def} \PY{n+nf}{generate\PYZus{}palindromes}\PY{p}{(}\PY{p}{)}\PY{p}{:}
        \PY{n}{ret} \PY{o}{=} \PY{p}{[}\PY{p}{]}
        \PY{n}{longest} \PY{o}{=} \PY{n}{N}
        \PY{n}{found} \PY{o}{=} \PY{k+kc}{False}

        \PY{k}{if} \PY{o+ow}{not} \PY{n}{strng}\PY{p}{:}
            \PY{k}{return} \PY{p}{[}\PY{l+s+s1}{\PYZsq{}}\PY{l+s+s1}{\PYZsq{}}\PY{p}{]}

        \PY{k}{for} \PY{n}{l} \PY{o+ow}{in} \PY{n+nb}{range}\PY{p}{(}\PY{n}{N}\PY{p}{,} \PY{l+m+mi}{0}\PY{p}{,} \PY{o}{\PYZhy{}}\PY{l+m+mi}{1}\PY{p}{)}\PY{p}{:}
            \PY{n}{found} \PY{o}{=} \PY{k+kc}{False}
            \PY{k}{for} \PY{n}{s} \PY{o+ow}{in} \PY{n+nb}{range}\PY{p}{(}\PY{n}{N}\PY{o}{\PYZhy{}}\PY{n}{l}\PY{o}{+}\PY{l+m+mi}{1}\PY{p}{)}\PY{p}{:}
                \PY{k}{if} \PY{n}{is\PYZus{}palindrome}\PY{p}{(}\PY{n}{s}\PY{p}{,} \PY{n}{s}\PY{o}{+}\PY{n}{l}\PY{o}{\PYZhy{}}\PY{l+m+mi}{1}\PY{p}{)}\PY{p}{:}
                    \PY{n}{found} \PY{o}{=} \PY{k+kc}{True}
                    \PY{n}{ret}\PY{o}{.}\PY{n}{append}\PY{p}{(}\PY{n}{strng}\PY{p}{[}\PY{n}{s}\PY{p}{:}\PY{n}{s}\PY{o}{+}\PY{n}{l}\PY{p}{]}\PY{p}{)}
            \PY{k}{if} \PY{n}{found}\PY{p}{:}
                \PY{k}{break}
        \PY{k}{return} \PY{n}{ret}

    \PY{k}{return} \PY{n}{generate\PYZus{}palindromes}\PY{p}{(}\PY{p}{)}
\end{Verbatim}
\end{tcolorbox}

    \begin{tcolorbox}[breakable, size=fbox, boxrule=1pt, pad at break*=1mm,colback=cellbackground, colframe=cellborder]
\prompt{In}{incolor}{197}{\boxspacing}
\begin{Verbatim}[commandchars=\\\{\}]
\PY{n}{longest\PYZus{}palindrom} \PY{o}{=} \PY{p}{[}\PY{p}{]}
\PY{c+c1}{\PYZsh{}look for the palindorms in each book}
\PY{k}{for} \PY{n}{title}\PY{p}{,}\PY{n}{book} \PY{o+ow}{in} \PY{n+nb}{zip}\PY{p}{(}\PY{n}{books}\PY{p}{,}\PY{n}{parsed\PYZus{}books}\PY{p}{)}\PY{p}{:}
    \PY{n+nb}{print}\PY{p}{(}\PY{l+s+s2}{\PYZdq{}}\PY{l+s+s2}{Parsing }\PY{l+s+si}{\PYZob{}\PYZcb{}}\PY{l+s+s2}{ .... \PYZsh{} of sentences }\PY{l+s+si}{\PYZob{}\PYZcb{}}\PY{l+s+s2}{\PYZdq{}}\PY{o}{.}\PY{n}{format}\PY{p}{(}\PY{n}{title}\PY{p}{,}\PY{n+nb}{len}\PY{p}{(}\PY{n}{parsed\PYZus{}books}\PY{p}{[}\PY{l+m+mi}{0}\PY{p}{]}\PY{o}{.}\PY{n}{sentences}\PY{p}{)}\PY{p}{)}\PY{p}{)}
    \PY{n}{li} \PY{o}{=} \PY{p}{[}\PY{p}{]} \PY{c+c1}{\PYZsh{}store the longest palindrome of each sentence.}
    
    \PY{n}{table} \PY{o}{=} \PY{n+nb}{str}\PY{o}{.}\PY{n}{maketrans}\PY{p}{(}\PY{l+s+s1}{\PYZsq{}}\PY{l+s+s1}{\PYZsq{}}\PY{p}{,} \PY{l+s+s1}{\PYZsq{}}\PY{l+s+s1}{\PYZsq{}}\PY{p}{,} \PY{n}{string}\PY{o}{.}\PY{n}{punctuation}\PY{o}{+}\PY{l+s+s1}{\PYZsq{}}\PY{l+s+s1}{‐}\PY{l+s+s1}{\PYZsq{}}\PY{p}{)}
    \PY{n}{stripped} \PY{o}{=} \PY{p}{[}\PY{l+s+s1}{\PYZsq{}}\PY{l+s+s1}{ }\PY{l+s+s1}{\PYZsq{}}\PY{o}{.}\PY{n}{join}\PY{p}{(}\PY{n}{w}\PY{o}{.}\PY{n}{translate}\PY{p}{(}\PY{n}{table}\PY{p}{)}\PY{o}{.}\PY{n}{split}\PY{p}{(}\PY{p}{)}\PY{p}{)} \PY{k}{for} \PY{n}{w} \PY{o+ow}{in} \PY{n}{book}\PY{o}{.}\PY{n}{sentences}\PY{p}{]}
    \PY{k}{for} \PY{n}{idx}\PY{p}{,} \PY{n}{sentence} \PY{o+ow}{in} \PY{n+nb}{enumerate}\PY{p}{(}\PY{n}{stripped}\PY{p}{)}\PY{p}{:}
        
        \PY{n}{li}\PY{o}{.}\PY{n}{append}\PY{p}{(}\PY{n}{get\PYZus{}longest\PYZus{}palindromes}\PY{p}{(}\PY{n}{sentence}\PY{p}{)}\PY{p}{)}
        
    \PY{n}{longest\PYZus{}palindrom}\PY{o}{.}\PY{n}{append}\PY{p}{(}\PY{n+nb}{max}\PY{p}{(}\PY{n}{li}\PY{p}{,} \PY{n}{key}\PY{o}{=} \PY{k}{lambda} \PY{n}{x}\PY{p}{:} \PY{n+nb}{len}\PY{p}{(}\PY{n}{x}\PY{p}{[}\PY{l+m+mi}{0}\PY{p}{]}\PY{p}{)}\PY{p}{)}\PY{p}{)}
        
    
\end{Verbatim}
\end{tcolorbox}

    \begin{Verbatim}[commandchars=\\\{\}]
Parsing Great Expectations {\ldots} \# of sentences 7245
Parsing Pride\_and\_Prejudice {\ldots} \# of sentences 7245
Parsing Pygmalion by Bernard Shaw {\ldots} \# of sentences 7245
Parsing The Brothers Karamazov by Fyodor Dostoyevsky {\ldots} \# of sentences 7245
Parsing The\_Adventures\_of\_Tom\_Sawyer {\ldots} \# of sentences 7245
Parsing Treasure Island by Robert Louis Stevenson {\ldots} \# of sentences 7245
    \end{Verbatim}

    \begin{tcolorbox}[breakable, size=fbox, boxrule=1pt, pad at break*=1mm,colback=cellbackground, colframe=cellborder]
\prompt{In}{incolor}{201}{\boxspacing}
\begin{Verbatim}[commandchars=\\\{\}]
\PY{k}{for} \PY{n}{title}\PY{p}{,}\PY{n}{palindrom} \PY{o+ow}{in} \PY{n+nb}{zip}\PY{p}{(}\PY{n}{books}\PY{p}{,}\PY{n}{longest\PYZus{}palindrom}\PY{p}{)}\PY{p}{:}
    
    \PY{n+nb}{print}\PY{p}{(}\PY{l+s+s2}{\PYZdq{}}\PY{l+s+s2}{The longest palindrome in: }\PY{l+s+si}{\PYZob{}\PYZcb{}}\PY{l+s+s2}{, is: \PYZhy{}\PYZhy{}\PYZgt{}}\PY{l+s+si}{\PYZob{}\PYZcb{}}\PY{l+s+s2}{\PYZlt{}\PYZhy{}\PYZhy{}}\PY{l+s+s2}{\PYZdq{}}\PY{o}{.}\PY{n}{format}\PY{p}{(}\PY{n}{title}\PY{p}{,}\PY{n}{palindrom}\PY{p}{[}\PY{l+m+mi}{0}\PY{p}{]}\PY{p}{)}\PY{p}{)}
\end{Verbatim}
\end{tcolorbox}

    \begin{Verbatim}[commandchars=\\\{\}]
The longest palindrome in: Great Expectations, is: -->iced a deci<--
The longest palindrome in: Pride\_and\_Prejudice, is: -->on did no<--
The longest palindrome in: Pygmalion by Bernard Shaw, is: -->aaaaaaaaa<--
The longest palindrome in: The Brothers Karamazov by Fyodor Dostoyevsky, is:
-->oorooroorooroo<--
The longest palindrome in: The\_Adventures\_of\_Tom\_Sawyer, is: -->id I di<--
The longest palindrome in: Treasure Island by Robert Louis Stevenson, is: -->
saw was <--
    \end{Verbatim}

    \hypertarget{comment}{%
\paragraph{Comment}\label{comment}}

 To compute the Palindrome, we divied the text into a set of sentences,
then we look for Palindromes an each of these sentences. This motivation
for this approach is that Palindromes have very high probability to
appear with each sentence not within sentences.

The function \texttt{get\_longest\_palindromes}, look for the longest
Palindrom in each sentence then. 

    \hypertarget{part-e-conclusion}{%
\subsection{Part E: Conclusion}\label{part-e-conclusion}}

     Few conclusion can be made using the results that we got above.

If we look at the histogram plots, one can see that the authors of the
books \texttt{The\ Adventures\ of\ Tom\ Sawyer\ by\ Mark\ Twain} and
\texttt{Great\ Expectations\ by\ Charles\ Dickensand} tend to write long
sentences while the first have a high pattern of writing long words and
the former is more like of a short words author.

Looking at the number of unique words in each book, show that the
authors \texttt{The\ Brothers\ Karamazov\ by\ Fyodor\ Dostoyevsky},
\texttt{The\ Adventures\ of\ Tom\ Sawyer\ by\ Mark\ Twain}, and
\texttt{Great\ Expectations\ by\ Charles\ Dickens} tend to have richer
vocabulary. The two former authors have quite similar vocabulary length.
Now, looking at the unique words in each book compared to the other
books, shows that the author of
\texttt{The\ Brothers\ Karamazov\ by\ Fyodor\ Dostoyevsky} uses far more
different vocabulary then the other authors. 


    % Add a bibliography block to the postdoc
    
    
    
\end{document}
